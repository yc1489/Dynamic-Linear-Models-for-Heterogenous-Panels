\documentclass[12pt,a4paper,hyperref]{article}
\usepackage[usenames,dvipsnames]{xcolor}
\definecolor{darkblue}{rgb}{0.0, 0.0, 0.55}
	\definecolor{ultramarine}{rgb}{0.07, 0.04, 0.56}
\usepackage{amsmath, natbib, latexsym, array, amssymb,longtable,float, graphicx, appendix,lscape,diagbox,textcomp,placeins}
\usepackage[colorlinks,
            linkcolor=ultramarine,
            anchorcolor=green,
            citecolor=darkblue
            ]{hyperref}
\usepackage[flushleft]{threeparttable}
\usepackage[top=2.7cm, left=3cm, right=3cm, bottom=2.7cm]{geometry}
\usepackage{hyperref}
\hypersetup{
    colorlinks=true,
    linkcolor=blue,
    filecolor=magenta,
    urlcolor=cyan,
}

\urlstyle{same}
\usepackage{booktabs}
\usepackage{siunitx}
\usepackage{pgfplotstable}
\sisetup{
  round-mode          = places, % Rounds numbers
  round-precision     = 2, % to 2 places
}

\newenvironment{sequation}{\begin{equation}\tiny}{\end{equation}}
\DeclareMathOperator*{\plim}{plim}
\renewcommand{\floatpagefraction}{0.60}
\renewcommand{\appendixpagename}{\Large Appendix}
\setcounter{secnumdepth}{3}
\begin{document}
\begin{titlepage}

\newcommand{\HRule}{\rule{\linewidth}{0.5mm}} % Defines a new command for the horizontal lines, change thickness here

\center % Center everything on the page



\HRule \\[0.4cm]
{ \huge \bfseries Monte Carlo simulation report }\\[0.4cm] % Title of your document
\HRule \\[1.5cm]


\vfill % Fill the rest of the page with whitespace

\end{titlepage}

\newpage
\tableofcontents
\newpage
\section{Monte Carlo simulation design}
\subsection{dynamic heterogeneous panels data model without error factor structure}
The data generating process:
\begin{align}
\begin{split}
y_{i,t}&= \phi_{i} y_{i,t-1}+ \sum^{k}_{\ell=1}\beta_{\ell i}x_{\ell i,t}+u_{i,t}, \,\, for\,\,i=1,\ldots N;\,t=-49,\ldots,T\, , \\ \label{M1}
x_{\ell i,t}&=\sum^{k}_{\ell=1}\phi_{\ell i}x_{\ell i,t-1}+v_{\ell i,t},
\end{split}
\end{align}
where $u_{i,t}\sim \mathcal{N}(0,\,1)\,,$ and $v_{\ell i,t}=\rho_{v, \ell}v_{\ell i,t-1}+\left( 1-\rho^{2}_{v, \ell} \right)^{\frac{1}{2}}\varpi_{\ell i,t}, \varpi_{\ell i,t} \overset{i.i.d.}{\sim} U(0.5, 1.5) \,.$

The slope coefficients are generated as 
\begin{align}
\phi_{i}=\phi+\eta_{\phi i},\,\, \beta_{1,i}=\beta_{1}+\eta_{\beta_{1} i}\, and\, \beta_{2,i}=\beta_{2}+\eta_{\beta_{2}i}.
\end{align}
Here we consider $\phi \in \left\{0.5, 0.8 \right\}$, $\beta_{1}=3$ and $\beta_{2}=1$. For the design of heterogenous slopes, $\eta_{\phi i} \overset{i.i.d.}{\sim} U\left( -c, c\right)$, and 
\begin{align}
\eta_{\beta_{\ell}i}= \left(1-\rho^{2}_{\beta}  \right)^{1/2}\eta_{\phi i}. 
\end{align}
Here, we set $c=0.2,\, \rho_{\beta}=0.4$ for $\ell=1,2.$


\subsection{Dynamic heterogeneous panels data model with multi-factor error structure}
This Monte Carlo simulation design same as \citet{Norkute:2019}. 
For convenience, we rewrite the data generating process as bellow
\begin{align}
y_{i,t}=\alpha_{i}+\phi_{i} y_{i,t-1}+ \sum^{k}_{\ell=1}\beta_{\ell i}x_{\ell i,t}+u_{i,t}, \,\, for\,\,i=1,\ldots N;\,t=-49,\ldots,T\, . \\ \label{M1}
\end{align}
We allow error factor structure in the model as
\begin{align}
u_{i,t}=    \sum^{m_{y}}_{s=1}\gamma^{0}_{si}f^{0}_{s,t} +\varepsilon_{i,t},
\end{align}
where 
\begin{align}
f^{0}_{s,t}=\rho^{0}_{s,t} f^{0}_{s,t-1}+\left( 1-\rho^{2}_{fs} \right)^{1/2}\zeta_{s,t},
\end{align}
with $\zeta_{s,t} \overset{i.i.d.}{\sim} N(0,1)$ for $s=1,\ldots m_{y}$. We assume $k=2$ and $m_{y}=1+k=3$ and set $\rho^{0}_{s,t}=0.5$ for all $s$. The error term, $\varepsilon_{i,t}$, setting as
\begin{align}
\varepsilon_{i,t}=\varsigma_{\varepsilon}\sigma_{it}\left(\epsilon_{it}-1 \right)/\sqrt{2},
\end{align}
where $ \epsilon_{it}\overset{i.i.d.}{\sim} \chi^{2}_{1},\, \sigma^{2}_{it}=\eta_{i}\varphi_{t}, \,\eta_{i}\overset{i.i.d.}{\sim}\chi^{2}_{2}/2,$  and  $\varphi_{t}=t/T\, for \,\, t=0,\ldots, T.$ And we set 
\begin{align}
\varsigma_{\varepsilon}=\frac{\pi_{\mu}}{1-\pi_{\mu}}m_{y}.
\end{align}
we set $\pi_{\mu} \in \left\{ 1/4, 3/4\right\}.$

The process of regressors is
\begin{align}
x_{\ell it}=\mu_{\ell i}++\sum^{k}_{\ell=1}\phi_{\ell i}x_{\ell i,t-1}+\sum^{m_{x}}_{s=1}\gamma^{0}_{\ell si}f^{0}_{s,t}+v_{\ell it}, \,\, for\,\,i=1,\ldots N;\,t=-49,\ldots,T\,; \, \ell=1,2.
\end{align}
We set number of factor, $m_{x}$, is $2$. Therefore, $\boldsymbol{f}^{0}_{y,t}=\left(f^{0}_{1t}, f^{0}_{2t},f^{0}_{3t}  \right)^{'}$ and $\boldsymbol{f}^{0}_{x,t}=\left(f^{0}_{1t}, f^{0}_{2t}  \right)^{'}.$
We set 
\begin{align}
v_{\ell i,t}=\rho_{v, \ell}v_{\ell i,t-1}+\left( 1-\rho^{2}_{v, \ell} \right)^{\frac{1}{2}}\varpi_{\ell i,t},  for \, \ell=1,2,
\end{align}
  where $\rho_{v, \ell}=0.5$ for all $\ell$.
The individual effect is
\begin{align}
 \alpha^{\ast}_{i}\overset{i.i.d.}{\sim} N\left(0, (1-\rho_{i})^{2} \right), \,\, \mu^{\ast}_{\ell i}=\rho_{\mu,\ell}\alpha^{\ast}_{i}+\left( 1-\rho^{2}_{\mu,\ell} \right)^{1/2}\omega_{\ell i},
\end{align}
where $\omega \overset{i.i.d.}{\sim}N\left(0, (1-\rho_{i})^{2} \right)$ and $\rho_{\mu,\ell}=0.5$.

Now, we define the factor loading in $u_{i,t}$ are generated as $\gamma^{0\ast}_{si}\overset{i.i.d.}{\sim}N\left(0, 1 \right)$, for $s=1,\ldots, m_{y}=3$, and the factor loading in $x_{1it}$ and $x_{2it}$ are drawn as
\begin{align}
\begin{split}
\gamma^{0\ast}_{1si}&=\rho_{\gamma,1s} \gamma^{0\ast}_{3i}+\left(1-\rho^{2}_{\gamma,1s}  \right)^{1/2}\xi_{1si};\,\xi_{1si}\overset{i.i.d.}{\sim}N\left(0, 1 \right); \\
\gamma^{0\ast}_{2si}&=\rho_{\gamma,2s} \gamma^{0\ast}_{si}+\left(1-\rho^{2}_{\gamma,2s}  \right)^{1/2}\xi_{2si};\,\xi_{2si}\overset{i.i.d.}{\sim}N\left(0, 1 \right);
\end{split}
\end{align}
for $s=1,\ldots, m_{x}=2$. We set $\rho_{\gamma, 11}=\rho_{\gamma, 12} \in \left\{0, 0.5 \right\}$ and $\rho_{\gamma, 21}=\rho_{\gamma, 22}=0.5.$
The factor loading are generated as
\begin{align}
\boldsymbol{\Gamma}=\boldsymbol{\Gamma}^{0}+\boldsymbol{\Gamma}^{0\ast}_{i}
\end{align}
where
\begin{align}
\boldsymbol{\Gamma}^{0}_{i}=
\begin{bmatrix}
\gamma^{0}_{1i} & \gamma^{0}_{11i} & \gamma^{0}_{21i} \\
\gamma^{0}_{2i} & \gamma^{0}_{12i} & \gamma^{0}_{22i} \\
\gamma^{0}_{3i} &  0               &     0
\end{bmatrix}
\end{align}
and 
\begin{align}
\boldsymbol{\Gamma}^{0\ast}_{i}=
\begin{bmatrix}
\gamma^{0\ast}_{1i} & \gamma^{0\ast}_{11i} & \gamma^{0\ast}_{21i} \\
\gamma^{0\ast}_{2i} & \gamma^{0\ast}_{12i} & \gamma^{0\ast}_{22i} \\
\gamma^{0\ast}_{3i} &  0               &     0
\end{bmatrix}.
\end{align}
We set 
\begin{align}
\boldsymbol{\Gamma}^{0}=
\begin{bmatrix}
1/4 & 1/4 & -1 \\
1/2 & -1  & 1/4 \\
1/2 & 0   & 0
\end{bmatrix} .
\end{align}
And 
\begin{align}
\alpha_{i}=\alpha+ \alpha^{\ast}_{i}, \, \mu_{\ell i}= \mu_{\ell}+\mu^{\ast}_{\ell i},
\end{align}
where $\alpha=1/2$, $\mu_{1}=1$, $\mu_{2}=-1/2$.

The slope coefficients are generated as
\begin{align}
\phi_{i}=\phi+\eta_{\phi i},\,\, \beta_{1,i}=\beta_{1}+\eta_{\beta_{1} i}\, and\, \beta_{2,i}=\beta_{2}+\eta_{\beta_{2}i}.
\end{align}
Here we consider $\phi \in \left\{0.5, 0.8 \right\}$, $\beta_{1}=3$ and $\beta_{2}=1$. For the design of heterogenous slopes, $\eta_{\phi i} \overset{i.i.d.}{\sim} U\left( -c, c\right)$, and
\begin{align}
\eta_{\beta_{\ell}i}=\left[(2c)^{2}/12 \right]\rho_{\beta}\xi_{\beta \ell i}+ \left(1-\rho^{2}_{\beta}  \right)^{1/2}\eta_{\phi i},
\end{align}
where
\begin{align}
\xi_{\beta \ell i}=\frac{\bar{v^{2}_{\ell i}}- \bar{v^{2}_{\ell}}}{\left[ N^{-1}\sum^{N}_{i=1} \left( \bar{v^{2}_{\ell i}}- \bar{v^{2}_{\ell }}\right)^{2} \right]^{1/2} },
\end{align}
with $\bar{v^{2}_{ell i}}=T^{-1}\sum^{T}_{t=1}v^{2}_{\ell i t}$, $\bar{v^{2}_{\ell}}=N^{-1} \sum^{N}_{i=1} \bar{v^{2}_{\ell i}}$, for $\ell=1,2.$
Here, we set $c=0.2,\, \rho_{\beta}=0.4$ for $\ell=1,2.$ And 
\begin{align}
\varsigma^{2}_{v}=\varsigma^{2}_{\varepsilon}\left[SNR-\frac{\rho^{2}_{v}}{1-\rho^{2}_{v}}   \right]\left(\frac{\beta^{2}_{1}+\beta^{2}_{2}}{1-\rho^{2}_{v}}  \right)^{-1},
\end{align} 

where $SNR=4$. For the $(T,N)$, we consider $T \in \left\{25, 50, 100, 200  \right\}$ and  $N \in \left\{25, 50, 100, 200  \right\}.$




\section{Monte Carlo simulation results}





































\newpage

\addcontentsline{toc}{section}{Reference}
\renewcommand\refname{References}
\bibliographystyle{chicago}
\bibliography{1}

\end{document}  \href{*}{*} 